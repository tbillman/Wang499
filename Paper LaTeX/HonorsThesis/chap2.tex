\section{Literature Review}
{\color{red}
	By the time most people would save up to buy a home outright, they likely would not have much of their remaining life to occupy it. Mortgages allow younger people to occupy homes by borrowing a large sum from a bank and paying it back with interest, typically in monthly payments across 30 years. However, by lending mortgages, banks expose themselves to considerable risk. Banks can lose large sums of money if borrowers can no longer make payments and their home value has depreciated. Due to this high risk, there has been considerable research conducted in an effort to better select quality mortgage borrower candidates. In recent years, the development of new statistical regression and classification techniques have proven to be more effective than traditional linear models \cite{Sirignano}. Additionally, due to the availability of large public datasets of loan level data, it has become an attractive dataset to look for significant relationships. 
	
	When a bank writes a home loan, they take on considerable risk. If the borrower makes all their payments, the bank can recoup their original investment with interest. However, if the borrower reneges on their end of the agreement there can be considerable financial consequences for the bank. Because of the high-risk, high-reward nature of loans there has been much research conducted attempting to model the behavior of these mortgages. These are a few of the papers that have done similar research to our project.}
One project uses contemporary machine learning techniques to model whether loans will carry out as planned, end in default, or end prepaid \cite{Deng}. Of all the techniques used, the most accurate model was random forest (RF) classification. This model could classify loans with 93\% accuracy. However, a loan that loses the banks tens of thousands of dollars due to a default, and a loan where the bank can fully recover its outstanding liabilities are considered the same. To remedy this, we analyze the financial impact on banks by opting for an NPV approach. Additionally, due to the accuracy of RF classification, we decided to use an RF model in our analysis.  
Another project involved an unprecedented dataset of 120 million prime and subprime mortgages from 1995 to 2014\cite{Sirignano}. After adding local economic metrics to the data, neural networks were used to predict how many loans would be end as either prepaid or default within random portfolios of thousands of loans. This research showed that neural networks considerably outperformed similar analysis using traditional logit techniques. This is particularly impactful for agencies that package and sell mortgage-backed securities as it improves their methods of choosing loans for their products. This is also a good indicator that machine learning algorithms will yield better predictions of NPV as compared to linear regression.


